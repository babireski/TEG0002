\documentclass{article}

\usepackage{listings}

\title{Implementação de grafo sobre o conjunto de dados da flor {\it Iris\/}}
\author{\sc{Elian Babireski \& Vinícios Bidin}}
\date{5 de outubro de 2022}

\begin{document}

    \maketitle
    
    \section{Introdução}
        \paragraph{} Trata-se de um projeto que tem como proposta a implementação de um programa que receba como entrada um conjunto de dados sobre a flor {\it Iris}. O conjunto de dados apresenta informações sobre as medidas de comprimento e largura da sépala e da pétala de cento e cinquenta amostas de três espécies distintas da planta.
    
        \begin{equation}
            \sqrt{\sum_{i = 1} ^ n {(p_i - q_i)} ^ 2}
        \end{equation}

        A lonjura entre os pontos no espaço quadridimensional devem, em seguida, ser normalizadas por meio da equação abaixo (2).
        \begin{equation}
            x' = \frac{x - \min(x)}{\max(x) - \min(x)}
        \end{equation}


    \section{Documentação}
        Na presente seção, será apresentada a documentação detalhada do código implementado.

    \section{Resultados}
        Na presente seção, serão apresentados os resultados obtidos pela equipe.
    
    \section{Compilação}
        \subsection{Dependências}
            \paragraph{} São dependências do projeto, a biblioteca {\it Graphviz} sendo utilizada a linguagem {\it Dot} para a visualização do grafo gerado, utiliza-se ainda o {\it Makefile} para compilação de todo o projeto, para que não seja necessário ficar compilando os arquivos de forma individual.
        
        \subsection{Compilação}
            \paragraph{} Uma vez no diretório do projeto, navegue para adentro, com a utilização do comando {\it cd trabalho/}, após feito, compile, utilizando {\it make}.
        
        \subsection{Execução}
            \paragraph{} Para executar, basta executar o arquivo de saída chamado {\it main}, utilizando o comando {\it ./main}. Para utilização, utilize as funcionalidades desejadas até que seja inserido um {\it 0} para encerramento do programa. Os arquivos gerados ao se utilizar opções {\it 4}, {\it 5}, {\it 6} do menu, são colocados no diretório {\it data/}. Para visualizar o grafo gerado, basta abrir o arquivo no formato {\it (Scalable Vector Graphics)}, chamado {\it graph.svg}.
    \section{Conclusão}
        Na presente seção, será feita a conclusão.

\end{document}