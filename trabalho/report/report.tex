\documentclass{article}

\title{Implementação de grafo sobre o conjunto de dados da flor {\it Iris\/}}
\author{\sc{Elian Babireski \& Vinícios Bidin}}
\date{5 de outubro de 2022}

\begin{document}

    \maketitle
    
    \section{Introdução}
        Trata-se de um projeto que tem como proposta a implementação de um programa que receba como entrada um conjunto de dados sobre a flor {\it Iris}. O conjunto de dados apresenta informações sobre as medidas de comprimento e largura da sépala e da pétala de cento e cinquenta amostas de três espécies distintas da planta.
    
        \begin{equation}
            \sqrt{\sum_{i = 1} ^ n (p_i - q_i) ^ 2}
        \end{equation}

        A lonjura entre os pontos no espaço quadridimensional devem, em seguida, ser normalizadas por meio da equação abaixo (2).
        \begin{equation}
            x' = \frac{x - \min(x)}{\max(x) - \min(x)}
        \end{equation}


    \section{Documentação}
        Na presente seção, será apresentada a documentação detalhada do código implementado.

    \section{Resultados}
        Na presente seção, serão apresentados os resultados obtidos pela equipe.
    
    \section{Conclusão}
        Na presente seção, será feita a conclusão.

\end{document}