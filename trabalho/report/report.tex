\documentclass{article}

\usepackage{listings}

\title{Implementação de grafo sobre o conjunto de dados da flor {\it Iris\/}}
\author{\sc{Elian Babireski \& Vinícios Bidin}}
\date{5 de outubro de 2022}

\begin{document}

    \maketitle

    \section{Introdução}
        \paragraph{} Trata-se de um projeto que tem como proposta a implementação de um programa que receba como entrada um conjunto de dados sobre a flor {\it Iris}. O conjunto de dados apresenta informações sobre as medidas de comprimento e largura da sépala e da pétala de cento e cinquenta amostas de três espécies distintas da planta.
		\par Em suma, cada amostra da planta deverá ser tratada como um ponto no espaço quadridimensional. Em seguida, deve-se calcular a distância euclidiana – conforme (1) – entre todos os pares de pontos.

        \begin{equation}
            \sqrt{\sum_{i = 1} ^ n {(p_i - q_i)} ^ 2}
        \end{equation}

        Calculadas as distâncias essas devem, em seguida, ser normalizadas por meio da equação abaixo (2). Por fim, deverá ser montado um grafo onde cada nodo corresponde a uma amostra, e uma aresta entre dois pontos existirá somente se a distância entre eles for menor ou igual a 0.3.
        \begin{equation}
            x' = \frac{x - \min(x)}{\max(x) - \min(x)}
        \end{equation}

		\section{Compilação}
			\subsection{Dependências}
				\paragraph{} São dependências do projeto, a biblioteca {\it Graphviz} sendo utilizada a linguagem {\it Dot} para a visualização do grafo gerado, utiliza-se ainda o {\it Makefile} para compilação de todo o projeto, para que não seja necessário ficar compilando os arquivos de forma individual.

			\subsection{Compilação}
				\paragraph{} Uma vez no diretório do projeto, navegue para adentro, compile, utilizando {\it make}.

			\subsection{Execução}
				\paragraph{} Para executar, basta executar o arquivo de saída chamado {\it main}, utilizando o comando {\it ./main}. Para utilização, utilize as funcionalidades desejadas até que seja inserido um {\it 0} para encerramento do programa. Os arquivos gerados ao se utilizar opções {\it 4}, {\it 5}, {\it 6} do menu, são colocados no diretório {\it data/}. Para visualizar o grafo gerado, basta abrir o arquivo no formato {\it (Scalable Vector Graphics)}, chamado {\it graph.svg}.

    \section{Documentação}
		\paragraph{} Abaixo encontra-se listada a documentação de todas as funções usadas na aplicação:
        \begin{itemize}
			\item \texttt{graph* create()}: cria e retorna um descritor da estrutura do grafo;
			\item \texttt{void addNode(graph *descriptor, sample sample)}: recebe um des- critor de grafo e os dados de uma amostra e, em seguida, adiciona a amostra como um novo nodo do grafo;
			\item \texttt{void addEdge(graph *descriptor, int head, int tail)}: recebe um descritor de grafo e o índice de dois nodos e cria uma aresta entre eles;
			\item \texttt{void destroy(graph *descriptor)}: desaloca todo o grafo da memória;
			\item \texttt{float distance(sample p, sample q)}: calcula a distância euclidiana entre duas amostras;
			\item \texttt{float normalize(float x, float maximum, float minimum)}: recebe como entrada um valor e o normaliza com base no máximo e mínimo também fornecidos;
			\item \texttt{void readTxt(graph *descriptor)}: carrega o grafo de um arquivo .txt;
			\item \texttt{void readCsv(graph *descriptor)}: carrega os dados das amostras de um arquivo .cvs e gera os nodos;
			\item \texttt{void link(graph *descriptor)}: gera uma tabela com as distâncias normalizadas entre as amostras e, em seguida, gera as arestas entre os nodos de acordo com um limite definido (0.3);
			\item \texttt{void print(graph* graph)}: imprime o grafo no terminal;
			\item \texttt{void saveTxt(graph* graph)}: salva o grafo em um arquivo de texto;
			\item \texttt{void graphviz(graph *descriptor)}: gera um arquivo .dot que representa o grafo;
			\item \texttt{void plot()}: gera um arquivo .svg do grafo;
			\item \texttt{float accuracy(graph *graph)}: gera a acurácia dado um grafo.
		\end{itemize}

    \section{Resultados}
        \paragraph{} Utilizado como limite o valor 0.3 percebeu-se que as {\it Setosas} quase formaram um grafo desconexo das demais plantas. As {\it Versicolors} e as {\it Virginicas}, entretanto, apresentaram muitas relações entre si. Após testes realizados pela equipe, concluiu-se que os resultados mais visualmente agradáveis ocorreram quando o limite estava ao redor de 0.15, porque assim o grupo das {\it Setosas} fica completamente desconexo do restante, e as outras duas espécies parecem um pouco mais separadas (embora ainda muito interconectadas).
		\par Disso, pode-se concluir que as {\it Setosas} são mais facilmente diferenciáveis pelas dimensões da pétala e sépala do que as outras duas espécies de {\it Iris}.

    \section{Acurácia}
        \paragraph{} Para se obter a acurácia de 96.2\% da classificação das raças dos dois grandes grupos de {\itÍris}, sendo estes o grupo das {\it Setosas} e das demais, utilizando como limite da distância euclidiana, o valor de 0.3, foi feita uma matriz de confusão, utilizando os {\it true positive}, {\it true negative} e os {\it false negative} e {\it false positive} em que esta é preenchida percorrendo o grafo com dois laços de repetição.
        \par A função {\it accuracy} é a responsável por fazer a leitura e caminhar sobre as arestas do grafo passado como argumento e retorna a acurácia, fazendo os devidos cálculos durante o caminhar sobre o grafo.
\end{document}
